\newpage
\thispagestyle{empty}
\setcounter{page}{2}

\vspace*{\fill}

\begin{abstract}
    \noindent
        Mikrobiologiczne ogniwa paliwowe (\acrshort{mfc}) oparte są
        na metabolizmie mikroorganizmów oddychających beztlenowo,
        które uwalniają elektrony na anodę.
        Z kolej urządzenia \acrshort{amfc} (algae \acrshort{mfc}) to
        klasa systemów \acrshort{mfc} wykorzystujących dodatkowo
        jednokomórkowe organizmy autotroficzne
        do generowania tlenu niezbędnego podczas utylizacji
        elektronów na powierzchni katody.
        Celem konstruowania tego typu systemów jest produkcja
        energii elektrycznej, jednakże generowane natężenia są
        stosunkowo niskie, z czego wynika konieczność sprzężenia
        produkcji energii elektrycznej z innymi procesami,
        takimi jak oczyszczanie ścieków, czy produkcja wartościowych substancji.
        W niniejszej pracy testowano wstępnie możliwość wykorzystania
        w \acrshort{amfc} szczepów \textit{R. sphaeroides} oraz
        \textit{Synechocystis sp.} PCC 6803
        (z heterologową ekspresją mikrocystynazy \acrshort{mlra})
        rosnących na pożywkach zawierających ścieki komunalne (\acrshort{ww}).
        Analiza wzrostu \textit{R. sphaeroides} wykazała, że komórki tego szczepu
        dobrze funkcjonują przy wysokich zawartościach ścieków w pożywce.
        Wzrost został zahamowany dopiero przy stężeniach \acrshort{ww} ok.\ 80~\%,
        niezależnie od rodzaju stosowanych \acrshort{ww}\@.
        Z kolei aktywność \acrshort{mlra} w hodowli \textit{Synechocystis sp.} PCC 6803
        była nawet wyższa dla próbek z \acrshort{ww} względem próby kontrolnej,
        co mogło być spowodowane zawartością związków organicznych oraz metabolitów
        wtórnych indukujących produkcję białek i samego \acrshort{mlra}.
        Wyniki pomiarów napięcia w zaprojektowanym \acrshort{amfc} wskazują,
        że optymalna gęstość zawiesiny \textit{R. sphaeroides} względem zastosowanej
        anody powinna mieścić się w zakresie \acrshort{od}\textsubscript{600} 0.2-0.5\@.
        Dalszy wzrost gęstości zawiesiny prowadzi najprawdopodobniej
        do wzrostu oporu wewnętrznego i spadków napięcia.
        W dalszej perspektywie będą prowadzone analizy funkcjonowania zaprojektowanego
        \acrshort{amfc} pod kątem poziomu generowania energii elektrycznej,
        produkcji \acrshort{mlra} oraz bioremediacji nutrientów zawartych w ściekach.
\end{abstract}

\vspace{\fill}

\newpage
\thispagestyle{empty}

\vspace*{\fill}

\begin{otherlanguage}{british}
    \begin{abstract}
        \noindent
        Microbial fuel cells (\acrshort{mfc}) are based on metabolism
        of anoxic microorganisms that use anode as a final electrone acceptor.
        In turn, \acrshort{amfc} (algae \acrshort{mfc}) is a class of
        \acrshort{mfc} systems additionally utilizing
        autotrophic microorganisms to generate molecular oxygen
        vital for electrone disposal on the cathode surface.
        The goal of constructing this kind of systems is mainly the generation
        of electrical energy, however the power output is rather low.
        Therefore such an electricity production should be coupled
        with other processes such as a wastewater treatment
        or a production of valuable substances.
        In this thesis the possibility of emloyment (in \acrshort{amfc})
        of \textit{R. sphaeroides} and \textit{Synechocystis sp.}
        PCC 6803 (with heterologous expression of microcystinase \acrshort{mlra})
        growing on media containing municipal wastewater
        (\acrshort{ww}) was tested.
        Growth analysis of \textit{R. Sphaeroides} showed that
        cells of this strain are thriving and functioning quite well
        at high concentrations of wastewater in medium.
        Growth would only be inhibited at concentrations of \acrshort{ww}
        about 80~\%, independent of wastewater type.
        Furthermore, \acrshort{mlra} activity in \textit{Synechocystis sp.} PCC 6803
        McCormick 7 culture was even higher in media with \acrshort{ww}
        than in a control sample which may be related to
        the presence of organic compounds and secondary metabolites
        inducing the production of proteins and \acrshort{mlra} specifically.
        The voltage measurrment in the designed \acrshort{amfc} indicates
        that optimal density of \textit{R. sphaeroides} suspension
        toward applied anode should be in range of
        \acrshort{od}\textsubscript{600} 0.2-0.5\@.
        Further increase of cell density most likely leads to the
        increased internal resistance and deacresed voltage.
        Future prospects are to analyse functionality of the designed
        \acrshort{amfc} in terms of generated power outputs,
        \acrshort{mlra} production and bioremediation of
        nutrients from wastewater.
    \end{abstract}
\end{otherlanguage}

\vspace{\fill}