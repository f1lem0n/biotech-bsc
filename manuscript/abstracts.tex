\begin{abstract}
    \noindent
    \acrshort{mfc} są ogniwami elektrochemicznymi wykorzystującymi
    mikroorganizmy do generowania elektronów uwalnianych na anodę.
    Urządzenia \acrshort{amfc} to klasa systemów \acrshort{mfc}
    wykorzystujących jednokomórkowe organizmy autotroficzne do
    generowania tlenu niezbędnego podczas utylizacji elektronów
    na powierzchni katody.
    W niniejszej pracy badano wzrost \textit{R. sphaeroides}
    na pożywkach zawierających ścieki komunalne (\acrshort{ww}),
    aktywność mikrocystynazy \acrshort{mlra} produkowanej
    przez komórki \textit{Synechocystis sp.}
    PCC 6803 McCormick 7 kultywowane na pożywkach zawierających
    \acrshort{ww} oraz zależności napięcia prądu stałego w układzie
    \acrshort{amfc} od gęstości zawiesiny \textit{R. sphaeroides}
    umieszczonej w komorze z anodą.
    Wzrost \textit{R. sphaeroides} został zahamowany dopiero
    przy stężeniach \acrshort{ww} ok.\ 80~\%, niezależnie
    od rodzaju stosowanych \acrshort{ww}.
    Wyniki pomiarów napięcia wskazują, że optymalna gęstość
    zawiesiny \textit{R. sphaeroides} względem zastosowanej anody powinna być
    w zakresie \acrshort{od}\textsubscript{600} 0.200-0.500.
    Dalszy wzrost gęstości zawiesiny prowadzi najprawdopodobniej
    do wzrostu oporu wewnętrznego i spadków napięcia.
    Aktywność mikrotcystynazy \acrshort{mlra} wzrosła dla
    próbek z \acrshort{ww} względem próby kontrolnej.
    Może to być spowodowane zawartością związków organicznych
    oraz metabolitów wtórnych indukujących produkcję białek
    i samego \acrshort{mlra}.
    Aby odpowiedzieć co dokładnie wpływa na produkcję
    \acrshort{mlra} należy przeprowadzić dalsze badania
    pod kątem analizy chemicznej oraz metabolomicznej
    badanych \acrshort{ww}.
\end{abstract}
%
\vspace{0.2cm}
%
\begin{center}
    \rule{150pt}{0.4pt}
\end{center}
%
\vspace{0.2cm}
%
\begin{otherlanguage}{british}
    \begin{abstract}
        \noindent
        \acrshort{mfc} is an electrochemical cell utilising
        microorganisms to generate electrons that are later
        realesed onto the anode surface. \acrshort{amfc}
        is a class of \acrshort{mfc} systems that uses autotrophic
        organisms to generate oxygen which is essential for
        electron disposal on the cathode surface.
        In the following thesis growth of \textit{R. spheroides}
        on media containing municipal wastewater (\acrshort{ww})
        was investigated as well as \acrshort{mlra} microcystinase
        activity produced by \textit{Synechocystis sp.} PCC 6803
        McCormick 7 cells growing on media containing \acrshort{ww},
        and the dependence of direct current voltage in \acrshort{amfc}
        system on \textit{R. sphaeroides} anode suspension's density.
        \textit{R. sphaeroides} growth was inhibited only by 80~\%
        \acrshort{ww} independent of \acrshort{ww} type.
        Voltage measurment results indicate that the optimal density
        of \textit{R. sphaeroides} suspension against applied anode
        should be in the range of \acrshort{od}\textsubscript{600}
        0.200-0.500.
        Further increase in the density most likely leads to
        the increase of internal resistance and therefore
        deacrease of voltage values.
        Microcystinase \acrshort{mlra} activity was significantly larger
        for samples containing \acrshort{ww} toward control sample.
        This could be caused by a presence of organic compounds and
        metabolites that might increase overall protein
        or mainly \acrshort{mlra} production.
        Further study focused mainly on chemical and metabolomic
        analysis of investigated \acrshort{ww} is needed in order
        to anwer what is the particular cause of increase in
        \acrshort{mlra} activity.
    \end{abstract}
\end{otherlanguage}