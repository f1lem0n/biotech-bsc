\documentclass[12pt, a4paper]{article}

%PACK------------------------------
\usepackage[utf8]{inputenc}
\usepackage[british, polish]{babel}
\usepackage{geometry}
\usepackage{gensymb} %pakiet symboli
\usepackage[upgreek, LGRgreek]{mathastext}
\usepackage{bm}
\usepackage{lipsum}
\usepackage{cite}
\usepackage{graphicx}
\usepackage{hyperref}
\usepackage{seqsplit}
\usepackage[monochrome]{xcolor}
\usepackage{mathptmx}
\usepackage{fontspec}
\usepackage{textcomp}

%SETTING---------------------------
%\defaultfontfeatures{LetterSpace=5}
\setmainfont{Times New Roman}
\setlength{\parindent}{0pt}
\setlength{\parskip}{0pt}
\linespread{1.25}
\geometry{left=2.54cm, right=2.54cm, top=2.54cm, bottom=2.54cm}
% \renewcommand{\thesection}{\arabic{section}.}
% \renewcommand{\thesubsection}{\thesection\arabic{subsection}.}
\newcommand*{\myfont}{\fontfamily{pcr}\selectfont}

%DOCUMENT--------------------------
\begin{document}
\begin{titlepage}
    \thispagestyle{empty}
    \begin{center}
    
        \textbf{\large Uniwersytet Jagielloński w Krakowie}
        
        \vspace{0.5cm}
        
        \textbf{\Large Wydział Biochemii, Biofizyki i Biotechnologii}
        
        \vspace{0.5cm}
        
        \begin{figure}[h]
            \centering
            \includegraphics[width=4cm]{figures/Herb_Uniwersytetu_Jagiellońskiego.svg.png}
            \label{fig:title}
        \end{figure}
        
        \vspace{0.5cm}

        \textbf{\LARGE Wstępna analiza funkcjonalności Mikrobiologicznego Ogniwa Paliwowego (MFC) ukierunkowanego na produkcję mikrocystynazy (MlrA) w \textit{Synechocystis sp.} PCC6803 na bazie ścieków komunalnych}

        \vspace{\fill}

%        \textbf{\large Preliminary analysis of the functionality of Microbial Fuel Cell (MFC) designed for microcystinase (MlrA) production in \textit{Synechocystis sp.} PCC6803 growing in municipal wastewater}

        \vspace{\fill}

        {\large Filip Stanisław Hajdyła}\\
        Nr albumu: 1164936

        \vspace{1.5cm}
        
        {\large Praca licencjacka z Biotechnologii}
        
        \vspace{0.3cm}
        
        {\large pod opieką dr hab. Dariusza Dzigi}
        
        \vspace{1cm}
        
        \textbf{\Large Pracownia Metabolomiki}
        
        \vspace{1cm}
        
        Kraków, 2022
        
    \end{center}
\end{titlepage}

\newpage
\thispagestyle{empty}
\begin{flushright}
    Dziękuję Panu Profesorowi Dariuszowi Dzidze\\
    oraz pozostałym członkom Pracowni Metabolomiki\\
    za cierpliwość, wyrozumiałość, cenne rady\\
    oraz pomoc w realizacji eksperymentów.
\end{flushright}
\vspace*{\fill}
\begin{flushleft}
    \textit{Non omnis moriar!}
\end{flushleft}


\newpage
\setcounter{page}{3}

\tableofcontents

\newpage
\thispagestyle{empty}
\setcounter{page}{2}

\vspace*{\fill}

\begin{abstract}
    \noindent
        Mikrobiologiczne ogniwa paliwowe (\acrshort{mfc}) oparte są
        na metabolizmie mikroorganizmów oddychających beztlenowo,
        które uwalniają elektrony na anodę.
        Z kolej urządzenia \acrshort{amfc} (algae \acrshort{mfc}) to
        klasa systemów \acrshort{mfc} wykorzystujących dodatkowo
        jednokomórkowe organizmy autotroficzne
        do generowania tlenu niezbędnego podczas utylizacji
        elektronów na powierzchni katody.
        Celem konstruowania tego typu systemów jest produkcja
        energii elektrycznej, jednakże generowane natężenia są
        stosunkowo niskie, z czego wynika konieczność sprzężenia
        produkcji energii elektrycznej z innymi procesami,
        takimi jak oczyszczanie ścieków, czy produkcja wartościowych substancji.
        W niniejszej pracy testowano wstępnie możliwość wykorzystania
        w \acrshort{amfc} szczepów \textit{R. sphaeroides} oraz
        \textit{Synechocystis sp.} PCC 6803
        (z heterologową ekspresją mikrocystynazy \acrshort{mlra})
        rosnących na pożywkach zawierających ścieki komunalne (\acrshort{ww}).
        Analiza wzrostu \textit{R. sphaeroides} wykazała, że komórki tego szczepu
        dobrze funkcjonują przy wysokich zawartościach ścieków w pożywce.
        Wzrost został zahamowany dopiero przy stężeniach \acrshort{ww} ok.\ 80~\%,
        niezależnie od rodzaju stosowanych \acrshort{ww}\@.
        Z kolei aktywność \acrshort{mlra} w hodowli \textit{Synechocystis sp.} PCC 6803
        była nawet wyższa dla próbek z \acrshort{ww} względem próby kontrolnej,
        co mogło być spowodowane zawartością związków organicznych oraz metabolitów
        wtórnych indukujących produkcję białek i samego \acrshort{mlra}.
        Wyniki pomiarów napięcia w zaprojektowanym \acrshort{amfc} wskazują,
        że optymalna gęstość zawiesiny \textit{R. sphaeroides} względem zastosowanej
        anody powinna mieścić się w zakresie \acrshort{od}\textsubscript{600} 0.2-0.5\@.
        Dalszy wzrost gęstości zawiesiny prowadzi najprawdopodobniej
        do wzrostu oporu wewnętrznego i spadków napięcia.
        W dalszej perspektywie będą prowadzone analizy funkcjonowania zaprojektowanego
        \acrshort{amfc} pod kątem poziomu generowania energii elektrycznej,
        produkcji \acrshort{mlra} oraz bioremediacji nutrientów zawartych w ściekach.
\end{abstract}

\vspace{\fill}

\newpage
\thispagestyle{empty}

\vspace*{\fill}

\begin{otherlanguage}{british}
    \begin{abstract}
        \noindent
        Microbial fuel cells (\acrshort{mfc}) are based on metabolism
        of anoxic microorganisms that use anode as a final electrone acceptor.
        In turn, \acrshort{amfc} (algae \acrshort{mfc}) is a class of
        \acrshort{mfc} systems additionally utilizing
        autotrophic microorganisms to generate molecular oxygen
        vital for electrone disposal on the cathode surface.
        The goal of constructing this kind of systems is mainly the generation
        of electrical energy, however the power output is rather low.
        Therefore such an electricity production should be coupled
        with other processes such as a wastewater treatment
        or a production of valuable substances.
        In this thesis the possibility of emloyment (in \acrshort{amfc})
        of \textit{R. sphaeroides} and \textit{Synechocystis sp.}
        PCC 6803 (with heterologous expression of microcystinase \acrshort{mlra})
        growing on media containing municipal wastewater
        (\acrshort{ww}) was tested.
        Growth analysis of \textit{R. Sphaeroides} showed that
        cells of this strain are thriving and functioning quite well
        at high concentrations of wastewater in medium.
        Growth would only be inhibited at concentrations of \acrshort{ww}
        about 80~\%, independent of wastewater type.
        Furthermore, \acrshort{mlra} activity in \textit{Synechocystis sp.} PCC 6803
        McCormick 7 culture was even higher in media with \acrshort{ww}
        than in a control sample which may be related to
        the presence of organic compounds and secondary metabolites
        inducing the production of proteins and \acrshort{mlra} specifically.
        The voltage measurrment in the designed \acrshort{amfc} indicates
        that optimal density of \textit{R. sphaeroides} suspension
        toward applied anode should be in range of
        \acrshort{od}\textsubscript{600} 0.2-0.5\@.
        Further increase of cell density most likely leads to the
        increased internal resistance and deacresed voltage.
        Future prospects are to analyse functionality of the designed
        \acrshort{amfc} in terms of generated power outputs,
        \acrshort{mlra} production and bioremediation of
        nutrients from wastewater.
    \end{abstract}
\end{otherlanguage}

\vspace{\fill}

\section{Wstęp teoretyczny}\label{sec:wstęp-teoretyczny}
\subsection{Historia}\label{subsec:historia}
MFC (\textit{ang.} Microbial Fuel Cell) to urządzenia umożliwiające
generowanie energii elektrycznej z wykorzystaniem mikroorganizmów.
Należą one do szerszej klasy urządzeń BES (\textit{ang.}
Bio-Electrochemical Systems)~\cite{Santoro2017}.
Pomysł wykorzystania mikroorganizmów do generowania elektryczności
przypisuje się Michaelowi Potterowi~\cite{Potter1911},
natomiast idea ,,elektryczności zwierząt'', a więc elektryczności
związanej z układami ożywionymi, sięga aż XVIII wieku~\cite{Santoro2017}.

\subsection{Zastosowania MFC}\label{subsec:zastosowania-mfc}
W ostatniej dekadzie zainteresowanie systemami BES,
a w szczególności MFC, wzrosło diametralnie, co odzwierciedla wzrost
liczby związanych z nimi publikacji przedstawiony na rys.~\ref{fig:1}.
Pojawiające publikacje związane z MFC dotyczą różnych aspektów tej technologii.
Wiele z nich skupia się na ogólnym zrozumieniu molekularnych podstaw
działania tych systemów~\cite{Slate2019, Bruce2006, Lovley2006}, inne
poruszają kwestie związane z ich konstrukcją i użyciem odpowiednich
materiałów do budowy elektrod i membran~\cite{Kaur2020, Daud2015},
a jeszcze inne skupiają się na bardzo istotnych aspektach ekonomicznych~\cite{Trapero2017}.
Wysokie zainteresowanie technologią MFC w ostatnich latach wynika
z potencjału do wykorzystania ich w celu jednoczesnego oczyszczania
wód ściekowych, generowania energii elektrycznej oraz cennej biomasy,
która mogłaby następnie zostać wykorzystana w biorafineriach do produkcji
biopaliw, biopolimerów czy biochemikaliów.

\begin{figure}[!b]
    \centering
    \includegraphics[width=\textwidth]{figures/publications}
    \caption{
        Liczba publikacji z ,,MFC'' lub
        ,,Microbial Fuel Cell'' w tytule w latach 1958--2021.
        (Dane pochodzą z serwisu \url{https://webofscience.com})
    }
    \label{fig:1}
\end{figure}

\subsection{Podstawy molekularne}\label{subsec:podstawy-molekularne}
W systemach MFC, do generowania elektryczności, wykorzystuje się
utleniająco-redukujący charakter reakcji metabolicznych
przeprowadzanych przez mikroorganizmy, które możemy podzielić
na rezydujące na powierzchni anody elektrogeny (uwalniające na nią
elektrony) oraz zasiedlające katodę elektrotrofy
(pobierające i wykorzystujące elektrony)~\cite{AlSayed2020}.
Elektrogeny przeprowadzają procesy oddychania beztlenowego,
utleniając znajdujące się w pożywce związki organiczne oraz
wykorzystując anodę jako ostateczne źródło elektronów.
Zdolność do transferu elektronów na powierzchnię metali (lub
innych przewodników) wynika z naturalnego przystosowania tych
organizmów do życia na powierzchni rud metali i wykorzystania
ich jako ostateczny akceptor elektronów.
Transfer elektronów na elektrodę może zachodzić na różne
sposoby~\cite{Santoro2017}:

\begin{enumerate}
    \item Transport bezpośredni (z wykorzystaniem cytochromu c);
    \item Transport przez nano-przewody;
    \item Transport za pośrednictwem mediatorów redox;
\end{enumerate}

Choć nie poznano jeszcze dokładnie molekularnych mechanizmów
transferu elektronów, wiadomo, że najwolniejszym oraz
niekorzystnym z technologicznego punktu widzenia sposobem jest
transport za pośrednictwem mediatorów redox, gdyż jest on znacznie
ograniczony szybkością dyfuzji.
Do najbardziej efektywnych elektrogenów należą
\textit{Geobacter sulfurreducens}, \textit{Shwanella oneidensis}
oraz \textit{Rhodobacter spheroides}.
Elektrotrofy są z kolei organizmami pobierającymi elektrony
z powierzchni katody (za pośrednictwem mediatorów redox)
i redukującymi CO$_2$, lub organizmami przeprowadzającymi fotosyntezę
oksygeniczną~\cite{Santoro2017, Reddy2019}.
Uwolniony w wyniku procesu fotosyntezy tlen jest następnie
wykorzystywany w reakcji redukcji
O$_2$~+~2H$^+$~+~2e$^-$~\textrightarrow~H$_2$O\@.
Użycie organizmów o wysokim tempie zużycia elektronów np.\ intensywnie
fotosyntetyzujących alg, takich jak \textit{Chlorella vulgaris},
lub cyjanobakterii z rodzaju \textit{Synechocystis} pozwala na efektywne
zapewnienie odpowiednich ilości tlenu w komorze katodowej bez
konieczności jej mechanicznego napowietrzania, co pozwala dodatkowo
zmniejszyć koszty operacyjne.
Należy jednak pamiętać, aby zapewnić odpowiednie warunki
bytującym w komorze organizmom.

\section{Metody}\label{sec:metody}

\subsection{metoda 1}
\lipsum[2]
\subsection{eksperyment uno}
\lipsum[3-5]

\section{Analiza Danych}\label{sec:analiza-danych}
\subsection{dane}
\lipsum[2-4]
\subsection{więcej danych}
\lipsum[2-4]
\subsection{jeszcze więcej zasranych danych}
\lipsum[2-4]

\begin{otherlanguage}{polish}
    \newpage
    \thispagestyle{empty}
    \bibliographystyle{unsrt}
    \bibliography{ref}
\end{otherlanguage}



\end{document}


