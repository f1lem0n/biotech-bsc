\subsection{Historia}\label{subsec:historia}
MFC (\textit{ang.} Microbial Fuel Cell) to urządzenia umożliwiające
generowanie energii elektrycznej z wykorzystaniem mikroorganizmów.
Należą one do szerszej klasy urządzeń BES (\textit{ang.}
Bio-Electrochemical Systems)~\cite{Santoro2017}.
Pomysł wykorzystania mikroorganizmów do generowania elektryczności
przypisuje się Michaelowi Potterowi~\cite{Potter1911},
natomiast idea ,,elektryczności zwierząt'', a więc elektryczności
związanej z układami ożywionymi, sięga aż XVIII wieku~\cite{Santoro2017}.

\subsection{Zastosowania MFC}\label{subsec:zastosowania-mfc}
W ostatniej dekadzie zainteresowanie systemami BES,
a w szczególności MFC, wzrosło diametralnie, co odzwierciedla wzrost
liczby związanych z nimi publikacji przedstawiony na rys.~\ref{fig:1}.
Pojawiające publikacje związane z MFC dotyczą różnych aspektów tej technologii.
Wiele z nich skupia się na ogólnym zrozumieniu molekularnych podstaw
działania tych systemów~\cite{Slate2019, Bruce2006, Lovley2006}, inne
poruszają kwestie związane z ich konstrukcją i użyciem odpowiednich
materiałów do budowy elektrod i membran~\cite{Kaur2020, Daud2015},
a jeszcze inne skupiają się na bardzo istotnych aspektach ekonomicznych~\cite{Trapero2017}.
Wysokie zainteresowanie technologią MFC w ostatnich latach wynika
z potencjału do wykorzystania ich w celu jednoczesnego oczyszczania
wód ściekowych, generowania energii elektrycznej oraz cennej biomasy,
która mogłaby następnie zostać wykorzystana w biorafineriach do produkcji
biopaliw, biopolimerów czy biochemikaliów.

\begin{figure}[!b]
    \centering
    \includegraphics[width=\textwidth]{figures/publications}
    \caption{
        Liczba publikacji z ,,MFC'' lub
        ,,Microbial Fuel Cell'' w tytule w latach 1958--2021.
        (Dane pochodzą z serwisu \url{https://webofscience.com})
    }
    \label{fig:1}
\end{figure}

\subsection{Podstawy molekularne}\label{subsec:podstawy-molekularne}
W systemach MFC, do generowania elektryczności, wykorzystuje się
utleniająco-redukujący charakter reakcji metabolicznych
przeprowadzanych przez mikroorganizmy, które możemy podzielić
na rezydujące na powierzchni anody elektrogeny (uwalniające na nią
elektrony) oraz zasiedlające katodę elektrotrofy
(pobierające i wykorzystujące elektrony)~\cite{AlSayed2020}.
Elektrogeny przeprowadzają procesy oddychania beztlenowego,
utleniając znajdujące się w pożywce związki organiczne oraz
wykorzystując anodę jako ostateczne źródło elektronów.
Zdolność do transferu elektronów na powierzchnię metali (lub
innych przewodników) wynika z naturalnego przystosowania tych
organizmów do życia na powierzchni rud metali i wykorzystania
ich jako ostateczny akceptor elektronów.
Transfer elektronów na elektrodę może zachodzić na różne
sposoby~\cite{Santoro2017}:

\begin{enumerate}
    \item Transport bezpośredni (z wykorzystaniem cytochromu c);
    \item Transport przez nano-przewody;
    \item Transport za pośrednictwem mediatorów redox;
\end{enumerate}

Choć nie poznano jeszcze dokładnie molekularnych mechanizmów
transferu elektronów, wiadomo, że najwolniejszym oraz
niekorzystnym z technologicznego punktu widzenia sposobem jest
transport za pośrednictwem mediatorów redox, gdyż jest on znacznie
ograniczony szybkością dyfuzji.
Do najbardziej efektywnych elektrogenów należą
\textit{Geobacter sulfurreducens}, \textit{Shwanella oneidensis}
oraz \textit{Rhodobacter spheroides}.
Elektrotrofy są z kolei organizmami pobierającymi elektrony
z powierzchni katody (za pośrednictwem mediatorów redox)
i redukującymi CO$_2$, lub organizmami przeprowadzającymi fotosyntezę
oksygeniczną~\cite{Santoro2017, Reddy2019}.
Uwolniony w wyniku procesu fotosyntezy tlen jest następnie
wykorzystywany w reakcji redukcji
O$_2$~+~2H$^+$~+~2e$^-$~\textrightarrow~H$_2$O\@.
Użycie organizmów o wysokim tempie zużycia elektronów np.\ intensywnie
fotosyntetyzujących alg, takich jak \textit{Chlorella vulgaris},
lub cyjanobakterii z rodzaju \textit{Synechocystis} pozwala na efektywne
zapewnienie odpowiednich ilości tlenu w komorze katodowej bez
konieczności jej mechanicznego napowietrzania, co pozwala dodatkowo
zmniejszyć koszty operacyjne.
Należy jednak pamiętać, aby zapewnić odpowiednie warunki
bytującym w komorze organizmom.