%RHODOBACTER
Wzrost \textit{R. sphaeroides}, badany poprzez pomiar
\acrshort{od}\textsubscript{600}, na pożywkach zawierających
\acrshort{ww} w~stężeniach nieprzekraczających 50~\%
nie został zahamowany w sposób istotny statystycznie.
Dopiero w przypadku 80~\% \acrshort{ww} wzrost był hamowany
po ok.\ 7-10 dniach hodowli.
Co ciekawe, nie było to zależne od rodzaju stosowanych
\acrshort{ww}, co wskazuje na niewrażliwość komórek
\textit{R.~sphaeroides} na różnego rodzaju zanieczyszczenia
i metabolity wtórne występujące w ściekach komunalnych
na różnych etapach oczyszczania, jeśli występują one
w umiarkowanych stężeniach.\\

%MLRA
Aktywność \acrshort{mlra} w przypadku komórek
\textit{Synechocystis spp.} PCC 6803 McCormick 7
wzrasta znacząco względem kontroli już po 4-7 dniach
prowadzenia hodowli.
Przyczyną tego zjawiska mogła być wyższa zawartość
związków organicznych w \acrshort{ww} wykorzystywanych
do produkcji białek lub zawartość metabolitów
wtórnych pochodzących od organizmów wykorzystywanych
do biodegradacji zanieczyszczeń na różnych etapach oczyszczania.
W celu zbadania różnic w~zawartości białka całkowitego
(będącego wskaźnikiem przyrostu biomasy)
wykreślono zależności zawartości białka w próbce od czasu
dla odpowiednich prób z eksperymentu~\ref{subsec:mlra}
(Dodatek, rys.~6).
Z wykreślonej zależności wynika, że zawartość białka
względem kontroli w ostatnich punktach pomiarowych
jest wyższa dla próbki zawierającej \acrshort{ww}3, natomiast
dla próbki z \acrshort{ww}2 różnica nie występuje.
Można więc wnioskować, że \acrshort{ww}3 są najprawdopodobniej
wzbogacone (względem BG-11) zarówno o związki organiczne podnoszące
wydajność produkcji biomasy, jak i o metabolity wtórne
pochodzenia mikrobiologicznego, które ukierunkowują metabolizm
\textit{Synechocystis spp.} PCC 6803 McCormick 7 na produkcję
mikrocystynazy \acrshort{mlra}. \acrshort{ww}2 natomiast
mogą być wzbogacone głównie o związki organiczne wykorzystywane
przez te komórki do produkcji białek ogólnie.
W celu zbadania, jakie związki organiczne i metabolity
wtórne wpływają na produkcję białka całkowitego
oraz w szczególności mikrocystynazy \acrshort{mlra},
potrzebne są dalsze badania ukierunkowane na analizy
chemiczne i metabolomiczne badanych próbek \acrshort{ww}.\\

%NAPIĘCIE
Początkowy wzrost napięcia w ogniwie \acrshort{amfc}
w zależności od \acrshort{od}\textsubscript{600}
\textit{R. sphaeroides} spowodowany jest wzrostem liczby
komórek uwalniających elektrony na powierzchnię anody,
generujących tym samym coraz wyższą różnicę potencjałów
pomiędzy elektrodami.
Późniejszy spadek napięcia wynika najpewniej z obecności
zbyt dużej liczby komórek \textit{R. sphaeroides}
względem ograniczonej powierzchni elektrody.
Wraz ze wzrostem liczby komórek rośnie ilość produkowanych
metabolitów, które mogą hamować procesy subkomórkowe oraz związków
wielkocząsteczkowych wchodzących w skład śluzu bakteryjnego,
który jest jedną z przyczyn wzrostu oporu wewnętrznego
i w konsekwencji spadku napięcia.