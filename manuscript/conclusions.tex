%RHODOBACTER
Wzrost \textit{R. sphaeroides}, badany poprzez pomiar
\acrshort{od}\textsubscript{600}, na pożywkach zawierających
\acrshort{ww} w~stężeniach nieprzekraczających 50~\%
nie został zahamowany w sposób istotny statystycznie.
Dopiero w przypadku 80~\% \acrshort{ww} wzrost był hamowany
po ok.\ 7-10 dniach hodowli.
Co ciekawe, nie było to zależne od rodzaju stosowanych
\acrshort{ww}, co wskazuje na niewrażliwość komórek
\textit{R.~sphaeroides} na różnego rodzaju zanieczyszczenia
i metabolity wtórne występujące w ściekach komunalnych
na różnych etapach oczyszczania, jeśli występują one
w umiarkowanych stężeniach.
%MLRA
%NAPIĘCIE
Początkowy wzrost napięcia w ogniwie \acrshort{amfc}
w zależności od \acrshort{od}\textsubscript{600}
\textit{R. sphaeroides} spowodowany jest wzrostem liczby
komórek uwalniających elektrony na powierzchnię anody,
generujących tym samym coraz wyższą różnicę potencjałów
pomiędzy elektrodami.
Późniejszy spadek napięcia wynika najpewniej z obecności
zbyt dużej liczby komórek \textit{R. sphaeroides}
względem ograniczonej powierzchni elektrody.
Wraz ze wzrostem liczby komórek rośnie ilość produkowanych
metabolitów, które mogą hamować metabolizm oraz związków
wielkocząsteczkowych wchodzących w skład śluzu bakteryjnego,
który jest jedną z przyczyn wzrostu oporu wewnętrznego
(wielkości mówiącej o spadku szybkości transferu elektronów
w obrębie i poza biofilm).
