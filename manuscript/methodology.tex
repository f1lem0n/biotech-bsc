\subsection{Sprzęt i oprogramowanie}\label{subsec:sprzet}
Do pomiarów gęstości optycznej (\acrshort{od}) używane były czytniki mikropłytek
Molecular Devices SpectraMax iD5, BioRad iMark
Microplate Reader, oraz spektrofotometr Unicam Helios $\upbeta$.
Do analizy aktywności \acrshort{mlra} metodą \acrshort{hplc} wykorzystano chromatograf
Agilent Technologies 1220 Infinity LC\@.
W celu homogenizacji komórek używano homogenizatora
Bertin Technologies Precellys Evolution (8000~rpm, 4 cykle po 45~s)
oraz kuleczek szklanych.
Celem pomiaru i ustalania pH~pożywek zastosowano pH-metr
BECKMAN $\Phi 50$ pH~Meter.
Do budowy układu \acrshort{amfc} wykorzystano oprzyrządowanie
zakupione w Mirong Mirong Store (Chiny) z~pojemnikami
o objętości 250~ml, membranę nafionową Sigma-Aldrich,
jako katodę -- płytkę mosiężną o powierzchni całkowitej 30~cm$^{2}$,
jako anodę -- 3 płytki z włókna węglowego
o~łącznej powierzchni całkowitej 84~cm$^{2}$
(warunki operacyjne układu:
$T =$ 20~\degree C, intensywność oświetlenia mierzona w miejscu
ustawienia butelek względem lampy była równa
70~$\upmu$mol fotonów m$^{-2}$~s$^{-1}$).
Do pomiarów różnicy potencjałów pomiędzy elektrodami
w układzie \acrshort{amfc} użyto multimetru PeakTech 4000 wraz
z~dedykowanym oprogramowaniem do zbierania danych przy pomocy komputera.
Do analizy danych i wizualizacji wykorzystano oprogramowanie
MS Excel, OriginLabs oraz język programowania Python
(głównie moduły pandas i matplotlib).

\subsection{Odczynniki chemiczne}\label{subsec:odczynniki}
Media hodowlane BG-11 i M27 do hodowli \textit{Rhodobacter} oraz
wszystkie potrzebne bufory wykonywane były zgodnie ze standardowymi
protokołami dostępnymi w Pracowni Metabolomiki.
Mikrocystynazę-LR pochodzi ze szczepu \textit{Microcystis aeruginosa} PCC 6803
pozyskanego z Instytutu Pasteura (Paryż, Francja).
Ścieki komunalne (\acrshort{ww}) pochodzą z oczyszczalni ścieków w Myślenicach.
Zostały one rozporcjowane, zamrożone, a przed użyciem wysterylizowane w~autoklawie.
Nie były one jednak poddane filtracji ani innym procesom oczyszczania.
\acrshort{ww} ponumerowano (1-4) wg.\ rys.~5 i tab.~1 (Dodatek).
Pozostałe odczynniki chemiczne zostały zakupione
w Merck Millipore (Burlington, MA, USA).

\subsection{Wykorzystane szczepy}\label{subsec:szczepy}
\textit{Synechocystis sp.} PCC 6803 McCormick 7~\cite{Puchalski2021}
był kultywowany z użyciem standardowego medium BG-11 z~dodatkiem 
50~$\upmu$l~ml$^{-1}$ kanamycyny, w kolbach Erlenmayera,
w temperaturze 28 \degree C i natężeniu światła
40~$\upmu$mol fotonów m$^{-2}$~s$^{-1}$.
Kolby były umieszczone na wytrząsarce orbitalnej (110~rpm).
\textit{Rhodobacter sphaeroides} kultywowano na pożywce \acrshort{m27}
do hodowli \textit{Rhodobacter} przygotowanej wg.\ protokołu
dostępnego w~Pracowni Metabolomiki, w probówkach typu
Falcon (50~ml), w temperaturze 20~\degree C i natężeniu światła
40~$\upmu$mol fotonów m$^{-2}$~s$^{-1}$.

\subsection{Badanie wzrostu \textit{R. sphaeroides} w pożywce z dodatkiem ścieków komunalnych}\label{subsec:rhodobacter}
W tym eksperymencie badano wzrost \textit{R. sphaeroides} przy
różnych rodzajach i rozcieńczeniach wód ściekowych w pożywce.
Eksperyment wykonano w dwóch turach.
W pierwszej turze, w~probówkach Falcon 15~ml sporządzono
mieszaniny pożywki \acrshort{m27} z~wodami ściekowymi nr 2, 3 i~4,
w~stężeniach 5~\%, 15~\% i~45~\% ($V$~=~14~ml).
Ustalono pH~mieszanin do ok.\ 7.
Do próby kontrolnej użyto pożywki \acrshort{m27} (pH~6.8).
Wszystkie próbki zaszczepiono 1~ml pierwotnej hodowli
\textit{R. sphaeroides}.
Pomiary \acrshort{od}\textsubscript{600} były wykonywane przy pomocy
spektrofotometru Unicam Helios $\upbeta$.
Drugą turę eksperymentu przygotowano w~sposób analogiczny.
Różnicą było testowanie innych stężeń wód ściekowych
(15~\%, 45~\% i~80~\%).
Hodowle wykonano w~trzech powtórzeniach biologicznych.
Mierzono \acrshort{od}\textsubscript{595} przy pomocy czytnika
mikropłytek BioRad iMark Microplate Reader
(droga optyczna $l$ odpowiada 200~$\upmu$l roztworu w studzience).

\subsection{Analiza aktywności~mlrA w \textit{Synechocystis sp.} PCC 6803 McCormick~7}\label{subsec:mlra}
Aktywność \acrshort{mlra} w komórkach
\textit{Synechocystis sp.} PCC 6803 McCormick 7 mierzono przy
różnych rodzajach \acrshort{ww} w pożywce.
W tym celu przygotowano mieszaniny BG-11 z \acrshort{ww}2 i \acrshort{ww}3
w proporcjach 1:1 w objętościach po 40~ml.
Kontrolą była hodowla prowadzona na pożywce BG-11.
Obliczono, że mając hodowlę macierzystą
o \acrshort{od}\textsubscript{600} = 0.746, w celu osiągnięcia
początkowego \acrshort{od}\textsubscript{600} = 0.200, należy zawiesić
komórki z odwirowanych (10733 RCF, 10~min) 10,72~ml hodowli
pierwotnej w 40~ml nowej pożywki.
Hodowle wykonano w trzech powtórzeniach biologicznych.
Markerem selekcyjnym była kanamycyna ($c$ = 50~$\upmu$g~ml$^{-1}$).
Zbierano próbki hodowli po 1~ml w 4, 7, 10 i~14 dniu eksperymentu
i komórki zamrażano (-20 \degree C).
Po rozmrożeniu próbki homogenizowano (\ref{subsec:sprzet}),
a~następnie odwirowywano (16873 RCF, 5~min), a~nadsącz
zbierano do nowych probówek typu Eppendorf 1.5~ml.
Na płytce 96-dołkowej przeprowadzono test aktywności \acrshort{mlra}\@.
Do studzienek dodano po 45~$\upmu$l mikrocystyny (1.5~mg~ml$^{-1}$)
i 5~$\upmu$l odpowiedniego, wcześniej zebranego nadsączu (lizatu)
w różnych rozcieńczeniach (1x i~10x).
Reakcję prowadzono przez 1 h, po czym zatrzymano ją
poprzez dodanie 5~$\upmu$l 1~\% \acrshort{tfa}\@.
Rozdział mieszaniny prowadzono przy pomocy techniki \acrshort{hplc} (\ref{subsec:sprzet}).
Celem normalizacji obliczonych na podstawie danych z \acrshort{hplc} aktywności białka
mierzono całkowite stężenie białka metodą Bradforda w 3 rozcieńczeniach:
\textbf{a} 5~$\upmu$l lizatu + 95~$\upmu$l buforu fosforanowego pH~7.0
+~100~$\upmu$l odczynnika Bradforda; \textbf{b} 2~$\upmu$l lizatu
+ 98~$\upmu$l buforu fosforanowego pH~7.0 +~100~$\upmu$l odczynnika;
\textbf{c} 1~$\upmu$l lizatu + 99~$\upmu$l buforu fosforanowego pH~7.0 +~100
$\upmu$l odczynnika.
Wykonano odpowiednie krzywe kalibracyjne.

\subsection{Pomiary napięcia w układzie aMFC}\label{subsec:volt}
W tym eksperymencie badano wpływ gęstości hodowli
\textit{R. sphaeroides} w komorze z anodą na generowane
w układzie \acrshort{amfc} napięcie.
W komorze z katodą znajdowały się komórki
\textit{Synechocystis sp.} PCC 6803 McCormick 7
zawieszone w 250~ml BG-11 + 50~\% \acrshort{ww}2
(\acrshort{od}\textsubscript{600} = 1.000).
Początkowo w komorze z anodą znajdowało się 250~ml
pożywki \acrshort{m27}, jednak na samej elektrodzie był
obecny wcześniej zaadaptowany biofilm \textit{R. sphaeroides}.
Dodawano po 1~ml stężonej zawiesiny \textit{R. sphaeroides}
i, po zmieszaniu, mierzono \acrshort{od}\textsubscript{600} oraz
zmiany napięcia prądu stałego (DC) przez ok.\ 200~min od
momentu dodania kolejnej porcji zawiesiny komórek.
Pomiary napięcia wykonywane były z częstotliwością 5 Hz.