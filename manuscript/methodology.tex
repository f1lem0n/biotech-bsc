\subsection{Sprzęt i oprogramowanie}\label{subsec:sprzet}
Do pomiarów absorbancji/OD używane były czytniki mikropłytek
Molecular Devices SpectraMax iD5, BioRad iMark
Microplate Reader, oraz spektrofotometr Unicam Helios $\upbeta$.
Do analizy aktywności MlrA metodą HPLC wykorzystano chromatograf
Agilent Technologies 1220 Infinity LC\@.
W celu homogenizacji komórek używano homogenizatora
Bertin Technologies Precellys Evolution oraz kuleczek szklanych.
Celem pomiaru i ustalania pH pożywek zastosowano pH-metr (\ldots).
Do pomiarów różnicy potencjałów pomiędzy elektrodami
w układzie MFC użyto multimetra PeakTech 4000 wraz
z~dedykowanym oprogramowaniem do zbierania danych
przy pomocy komputera.
Do analizy danych i wizualizacji wykorzystano oprogramowanie
MS Excel, OriginLabs oraz język programowania Python
(głównie moduły pandas i matplotlib).

\subsection{Odczynniki chemiczne}\label{subsec:odczynniki}
???

\subsection{Wykorzystane szczepy}\label{subsec:szczepy}
\textit{Synechocystis sp.} PCC 6803 były kultywowane z użyciem
standardowego medium BG-11 z~dodatkiem 50
$\upmu$l ml\textsuperscript{-1} kanamycyny, w kolbach Erlenmayera.
Używano konstruktów McCormick opisanych w tab.\ 1 (dodatek).
\textit{Rhodobacter spheroides} kultywowano w probówkach typu
Falcon (50 ml) na pożywce M27 do hodowli
\textit{Rhodobacter} przygotowanej wg.\ protokołu
dostępnego w~Pracowni Metabolomiki.

\subsection{Badanie wzrostu \textit{R. spheroides} na wodach ściekowych}\label{subsec:rhodobacter}

\subsection{Badanie aktywności MlrA w szczepach \textit{Synechocystis} McCormick}\label{subsec:mlra}

\subsection{Pomiary napięcia w układzie MFC}\label{subsec:volt}

\subsection{Produkcja MlrA w układzie MFC}\label{subsec:mfc}