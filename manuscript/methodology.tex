\subsection{Sprzęt i oprogramowanie}\label{subsec:sprzet}
Do pomiarów absorbancji/OD używane były czytniki mikropłytek
Molecular Devices SpectraMax iD5, BioRad iMark
Microplate Reader, oraz spektrofotometr Unicam Helios $\upbeta$.
Do analizy aktywności MlrA metodą HPLC wykorzystano chromatograf
Agilent Technologies 1220 Infinity LC\@.
W celu homogenizacji komórek używano homogenizatora
Bertin Technologies Precellys Evolution ((\ldots) rpm, 4 cykle po 45 s)
oraz kuleczek szklanych.
Celem pomiaru i ustalania pH pożywek zastosowano pH-metr (\ldots).
Do budowy układu MFC wykorzystano butelki firmy (\ldots)
o pojemności 250 ml, membranę nafionową (\ldots),
jako katodę — płytkę mosiężną o powierzchni całkowitej (\ldots)
cm\textsuperscript{2}, jako anodę — 3 płytki z włókna węglowego firmy
(\ldots) o łącznej powierzchni całkowitej (\ldots) cm\textsuperscript{2},
lampę o natężeniu 100 mol fotonów m\textsuperscript{-2} s\textsuperscript{-1}
Do pomiarów różnicy potencjałów pomiędzy elektrodami
w układzie MFC użyto multimetru PeakTech 4000 wraz
z~dedykowanym oprogramowaniem do zbierania danych przy pomocy komputera.
Do analizy danych i wizualizacji wykorzystano oprogramowanie
MS Excel, OriginLabs oraz język programowania Python
(głównie moduły pandas i matplotlib).

\subsection{Odczynniki chemiczne}\label{subsec:odczynniki}
Kwas trifluorooctowy (TFA), odczynnik Bradforda, bufor fosforanowy (pH 7),
media, wody ściekowe pochodzące z oczyszczalni w Myślenicach
(numeracja objaśniona w dodatku — tab.\ 1).

\subsection{Wykorzystane szczepy}\label{subsec:szczepy}
\textit{Synechocystis sp.} PCC 6803 McCormick 7~\cite{Puchalski2021}
były kultywowane z użyciem standardowego medium BG-11 z~dodatkiem 50
$\upmu$l ml\textsuperscript{-1} kanamycyny, w kolbach Erlenmayera,
w temperaturze 30 \degree C i natężeniu światła
40 $\upmu$mol fotonów m\textsuperscript{-2} min\textsuperscript{-1}.
\textit{Rhodobacter spheroides} kultywowano na pożywce M27
do hodowli \textit{Rhodobacter} przygotowanej wg.\ protokołu
dostępnego w~Pracowni Metabolomiki, w probówkach typu
Falcon (50 ml), w temperaturze (\ldots) \degree C i natężeniu światła
(\ldots) $\upmu$mol fotonów m\textsuperscript{-2} s\textsuperscript{-1}.

\subsection{Badanie wzrostu \textit{R. spheroides} na wodach ściekowych}\label{subsec:rhodobacter}
Eksperyment wykonano w dwóch turach.
W pierwszej turze, w probówkach Falcon 15 ml sporządzono
mieszaniny M27 z~wodami ściekowymi nr 2, 3 i 4,
w stężeniach 5 \%, 15 \% i~45~\% ($V$~=~14~ml).
Ustalono pH mieszanin do pH 7.
Jako kontroli użyto czystego M27 (pH 6.8).
Wszystkie próbki zaszczepiono 1 ml pierwotnej hodowli
\textit{R. spheroides}.
Pomiary OD\textsubscript{600} były wykonywane przy pomocy
spektrofotometru Unicam Helios $\upbeta$.
Drugą turę eksperymentu przygotowano w~sposób analogiczny.
Różnicą było testowanie innych stężeń wód ściekowych
(15~\%, 45 \% i~80 \%).
Próbki wykonano w tryplikatach.
Mierzono OD\textsubscript{595} przy pomocy czytnika
mikropłytek BioRad iMark Microplate Reader
(droga optyczna $l$ odpowiada 200 $\upmu$l roztworu w studzience).

\subsection{Badanie aktywności MlrA w \textit{Synechocystis} McCormick 7}\label{subsec:mlra}


\subsection{Pomiary napięcia w układzie MFC}\label{subsec:volt}

\subsection{Produkcja MlrA w układzie MFC}\label{subsec:mfc}