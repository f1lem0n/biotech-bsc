\subsection{Sprzęt i oprogramowanie}\label{subsec:sprzet}
Do pomiarów absorbancji/OD używane były czytniki mikropłytek
Molecular Devices SpectraMax iD5, BioRad iMark
Microplate Reader, oraz spektrofotometr Unicam Helios $\upbeta$.
Do analizy aktywności MlrA metodą HPLC wykorzystano chromatograf
Agilent Technologies 1220 Infinity LC\@.
W celu homogenizacji komórek używano homogenizatora
Bertin Technologies Precellys Evolution (8000 rpm, 4 cykle po 45 s)
oraz kuleczek szklanych.
Celem pomiaru i ustalania pH pożywek zastosowano pH-metr
BECKMAN $\Phi 50$ pH Meter.
Do budowy układu MFC wykorzystano butelki firmy (\ldots)
o pojemności 250 ml, membranę nafionową (\ldots),
jako katodę — płytkę mosiężną o powierzchni całkowitej (\ldots)
cm\textsuperscript{2}, jako anodę — 3 płytki z włókna węglowego firmy
(\ldots) o łącznej powierzchni całkowitej (\ldots) cm\textsuperscript{2}
(warunki operacyjne układu:
$T =$ 20 \degree C, intensywność oświetlenia mierzona w miejscu
ustawienia butelek względem lampy była równa
70 $\upmu$mol fotonów m\textsuperscript{-2} s\textsuperscript{-1}).
Do pomiarów różnicy potencjałów pomiędzy elektrodami
w układzie MFC użyto multimetru PeakTech 4000 wraz
z~dedykowanym oprogramowaniem do zbierania danych przy pomocy komputera.
Do analizy danych i wizualizacji wykorzystano oprogramowanie
MS Excel, OriginLabs oraz język programowania Python
(głównie moduły pandas i matplotlib).

\subsection{Odczynniki chemiczne}\label{subsec:odczynniki}
Kwas trifluorooctowy (TFA), odczynnik Bradforda, bufor fosforanowy (pH 7),
media, wody ściekowe (WW) pochodzące z oczyszczalni w Myślenicach
(numeracja objaśniona w dodatku — tab.\ 1).

\subsection{Wykorzystane szczepy}\label{subsec:szczepy}
\textit{Synechocystis sp.} PCC 6803 McCormick 7~\cite{Puchalski2021}
były kultywowane z użyciem standardowego medium BG-11 z~dodatkiem 50
$\upmu$l ml\textsuperscript{-1} kanamycyny, w kolbach Erlenmayera,
w temperaturze 28 \degree C i natężeniu światła
40 $\upmu$mol fotonów m\textsuperscript{-2} s\textsuperscript{-1}.
Kolbi były umieszczone na wytrząsarce orbitalnej (110 rpm).
\textit{Rhodobacter spheroides} kultywowano na pożywce M27
do hodowli \textit{Rhodobacter} przygotowanej wg.\ protokołu
dostępnego w~Pracowni Metabolomiki, w probówkach typu
Falcon (50 ml), w temperaturze 20 \degree C i natężeniu światła
40 $\upmu$mol fotonów m\textsuperscript{-2} s\textsuperscript{-1}.

\subsection{Badanie wzrostu \textit{R. spheroides} na wodach ściekowych}\label{subsec:rhodobacter}
W tym eksperymencie badano wzrost \textit{R. spheroides} przy
różnych rodzajach i rozcieńczeniach wód ściekowych w pożywce.
Eksperyment wykonano w dwóch turach.
W pierwszej turze, w~probówkach Falcon 15 ml sporządzono
mieszaniny M27 z~wodami ściekowymi nr 2, 3 i~4,
w~stężeniach 5 \%, 15 \% i~45~\% ($V$~=~14~ml).
Ustalono pH mieszanin do pH 7.
Jako kontroli użyto czystego M27 (pH 6.8).
Wszystkie próbki zaszczepiono 1 ml pierwotnej hodowli
\textit{R. spheroides}.
Pomiary OD\textsubscript{600} były wykonywane przy pomocy
spektrofotometru Unicam Helios $\upbeta$.
Drugą turę eksperymentu przygotowano w~sposób analogiczny.
Różnicą było testowanie innych stężeń wód ściekowych
(15~\%, 45 \% i~80 \%).
Hodowle wykonano w tryplikatach.
Mierzono OD\textsubscript{595} przy pomocy czytnika
mikropłytek BioRad iMark Microplate Reader
(droga optyczna $l$ odpowiada 200 $\upmu$l roztworu w studzience).

\subsection{Badanie aktywności MlrA w \textit{Synechocystis sp.} PCC 6803 McCormick~7}\label{subsec:mlra}
W tym eksperymencie mierzono aktywność MlrA w komórkach
\textit{Synechocystis sp.} PCC 6803 przy
różnych rodzajach WW w pożywce.
W tym celu przygotowano mieszaniny BG-11 z WW2 i WW3
w proporcjach 1:1 w objętościach po 40 ml.
Kontrolą była czysta pożywka BG-11.
Obliczono, że mając hodowlę macierzystą
o OD\textsubscript{600} = 0.746, w celu osiągnięcia
początkowego OD\textsubscript{600} = 0.200, należy zawiesić
komórki z odwirowanych (10733 RCF, 10 min) 10,72 ml hodowli
pierwotnej w 40 ml nowej pożywki.
Hodowle wykonano w tryplikatach.
Markerem selekcyjnym była kanamycyna ($c$ = 50 $\upmu$g ml\textsuperscript{-1}).
Zbierano próbki hodowli po 1 ml w 4, 7, 10 i~14 dniu eksperymentu
i zamrażano (-20 \degree C).
Po rozmrożeniu próbki homogenizowano (\ref{subsec:sprzet}),
a~następnie odwirowywano (16873 RCF, 5 min), a nadsącz
zbierano do nowych probówek typu Eppendorf 1.5 ml.
Na płytce 96-dołkowej przeprowadzono test aktywności MlrA\@.
Do studzienek dodano po 45 $\upmu$l mikrocystyny (1.5 mg ml\textsuperscript{-1})
i 5 $\upmu$l odpowiedniego, wcześniej zebranego nadsączu (lizatu).
Reakcję prowadzono przez 1 h, po czym zatrzymano ją
poprzez dodanie 5 $\upmu$l 1 \% TFA\@.
Rozdział mieszaniny prowadzono przy pomocy techniki HPLC (\ref{subsec:sprzet}).
Celem normalizacji obliczonych na podstawie danych z HPLC aktywności białka
mierzono całkowite stężenie białka metodą Bradforda:
5 $\upmu$l lizatu + 95 $\upmu$l buforu fosforanowego pH 7.0
+~100 $\upmu$l odczynnika Bradforda.

\subsection{Pomiary napięcia w układzie MFC}\label{subsec:volt}
W tym eksperymencie badano wpływ gęstości hodowli
\textit{R. spheroides} w komorze z anodą na generowane
w układzie MFC napięcie.
W komorze z katodą znajdowały się komórki
\textit{Synechocystis sp.} PCC 6803 McCormick 7
zawieszone w 250 ml BG-11 + 50 \% WW2
(OD\textsubscript{600} = 1.000).
Początkowo w komorze z anodą znajdowało się 250 ml
czystego M27, jednak na samej elektrodzie był
obecny wcześniej zaadaptowany biofilm \textit{R. spheroides}.
Dodawano po 1 ml stężonej zawiesiny \textit{R. spheroides}
i, po zmieszaniu, mierzono OD\textsubscript{600} oraz
zmiany napięcia w czasie do 220 min od momentu dodania
kolejnej porcji zawiesiny komórek.

\subsection{Produkcja MlrA w układzie MFC}\label{subsec:mfc}
W tym eksperymencie mierzono aktywność MlrA
w komórkach \textit{Synechocystis sp.} PCC 6803,
kultywowanych na BG-11 + 50 \% WW2 w układzie MFC\@.
W komorze z anodą znajdowały się komórki
\textit{R. spheroides} w M27 + 50 \% WW2,
o OD\textsubscript{600} = 0.200.
W komorze z katodą umieszczono zawiesinę komórek
\textit{Synechocystis sp.} PCC 6803 McCormick 7
w BG-11 + 50 \% WW2 (OD\textsubscript{600} = 0.200).
Tak przygotowane komórki kultywowano w układzie MFC (\ref{subsec:sprzet}).
W 1, 4, 6 i~8 dniu eksperymentu z hodowli zbierano próbki
celem analizy aktywności MlrA (\ref{subsec:mlra}).